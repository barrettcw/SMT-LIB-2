%!TEX root = main.tex

% basic-assumptions.tex

%===============================================================================
\chapter{Basic Assumptions and Structure} 
%===============================================================================
\thispagestyle{empty}


This chapter introduces the defining basic assumptions 
of the SMT-LIB standard and 
describes its overall structure.


\bigskip

\section{Satisfiability Modulo Theories}

The defining problem of Satisfiability Modulo Theories is checking 
whether a given (closed) logical formula $\varphi$ is \emph{satisfiable},
not in general but in the context of some background theory $\T$
which constrains the interpretation of the symbols used in $\varphi$.
Technically, the SMT problem for $\varphi$ and $\T$ is 
the question of 
whether there is a model of $\T$ that makes $\varphi$ true.
%For instance, \trem{example here}
 
A dual version of the SMT problem, 
which we could call \emph{Validity Modulo Theories},
asks whether a formula $\varphi$ is \emph{valid} in some theory $\T$,
that is, satisfied by every model of $\T$.
As the name suggests, SMT-LIB focuses only on the SMT problem.
However, at least for classes of formulas that are closed under logical negation,
this is no restriction because the two problems are inter-reducible:
a formula $\varphi$ is valid in a theory $\T$ exactly when 
its negation is not satisfiable in the theory.

Informally speaking, SMT-LIB calls an \define{SMT solver}\index{SMT!solver}
any software system that implements 
a procedure for satisfiability modulo some given theory.
In general,
one can distinguish among a solver's 
\begin{enumerate}
\item
\emph{underlying logic},
e.g., first-order, modal, temporal, second-order, and so on,

\item
\emph{background theory},
the theory against which satisfiability is checked,

\item
\emph{input formulas}, 
the class of formulas the solver accepts as input,
and

\item
\emph{interface}, 
the set of functionalities provided by the solver.
\end{enumerate}

For instance, in a solver for linear arithmetic
the underlying logic is first-order logic with equality,
the background theory is the theory of real numbers, and
the input language may be limited to conjunctions of inequations 
between linear polynomials.
The interface may be as simple as accepting a system
of inequations and returning a binary response indicating 
whether the system is satisfiable or not.
More sophisticated interfaces include
the ability to return concrete solutions for satisfiable inputs,
return proofs for unsatisfiable ones,
allow incremental and backtrackable input, and so on.

For better clarity and modularity, 
the aspects above are kept separate in SMT-LIB.
SMT-LIB's commitment to each of them is described in the following.


%-------------------------------------------------------------------------------
\section{Underlying Logic}
%-------------------------------------------------------------------------------


\new{The most recent past version of the SMT-LIB format (Version 2.6 )}
adopts as its underlying logic a version of many-sorted first-order logic 
with equality~\cite{Man-MSL-93,Gal-86,Hen-01}.
Like traditional many-sorted logic, it has sorts (i.e., basic types) 
and sorted terms.
Unlike that logic, however,
it does not have a syntactic category of formulas distinct from terms.
Formulas are just sorted terms of a distinguished Boolean sort,
which is interpreted as a two-element set in every SMT-LIB theory.\footnote{This is similar
to some formulations of classical higher-order logic, such as that of~\cite{andrews86}.}
Furthermore, the SMT-LIB logic uses a language of sort terms,
as opposed to just sort constants, to denote sorts:
sorts can be denoted by sort constants like \texttt{Int}
as well as sort terms like
\texttt{(List (Array Int Real))}.
Finally,
in addition to the usual existential and universal quantifiers, 
the logic includes a \emph{let} binder
and a \emph{match} binder
analogous to constructs with the same name
found in functional programming languages.

\begin{newver}
Version 2.7 introduces a new core theory of maps, 
values with a sort of the form \expr{(-> $\tau_1$ $\tau_2$)}, and 
extends the language of Version 2.6 with a $\lambda$ binder
for map abstraction,
as well as an explicit application operator \ter{\_}
for values of sort \expr{(-> $\tau_1$ $\tau_2$)}. 
This allows, in effect, the use of higher-order functions,
while keeping the underlying logic nominally first order.\footnote{
Version 3 of the standard, which will be based on a higher-order logic,
will identify maps and functions.
}
\end{newver}

SMT-LIB's underlying logic,
henceforth \define{SMT-LIB logic},
provides the formal foundations of the SMT-LIB standard.
The concrete syntax of the logic is part of the SMT-LIB language
of formulas and theories, which is defined in Part~\ref{part:syntax} 
of this document.
An abstract syntax for SMT-LIB logic and 
the logic's formal semantics are provided in Part~\ref{part:semantics}.



%-------------------------------------------------------------------------------
\section{Background Theories}
%-------------------------------------------------------------------------------

One of the goals of the SMT-LIB initiative is 
to clearly define a catalog of background theories,
starting with a small number of popular ones,
and adding new ones as solvers for them are developed.\footnote{
This catalog is available, separately from this document,
from the SMT-LIB website (\href{http://www.smt-lib.org}{www.smt-lib.org}).
}
Theories are specified in SMT-LIB independently of any benchmarks or solvers.
On the other hand, 
each SMT-LIB script refers, indirectly, to one or more theories 
in the SMT-LIB catalog.

This version of the SMT-LIB standard distinguishes 
between \emph{basic}\index{theory!basic} theories and 
\emph{combined}\index{theory!combined} theories.
Basic theories,
such as the theory of real numbers, the theory of arrays, 
the theory of fixed-size bit vectors and so on,
are those explicitly defined in the SMT-LIB catalog.
Combined theories are defined implicitly in terms of basic theories 
by means of a general modular combination operator.
The difference between a basic theory and a combined one
in SMT-LIB is essentially operational.
Some SMT-LIB theories, 
such as the theory of finite sets with a cardinality operator, 
are defined as basic theories,
even if they are in fact a combination of smaller theories,
because they cannot be obtained by modular combination.
%
%This document defines a standard format for specifying not single basic theories,
%but basic \emph{theory schemas},
%where a theory schema defines an \emph{infinite family} of basic theories 
%with similar features.
%For example, \trem{More}.

Theory specifications have mostly documentation purposes.
They are meant to be standard references for human readers.
For practicality then, the format insists that only the \emph{signature} of 
a theory (essentially, its set of sort symbols and sorted function symbols) 
be specified formally---provided it is finite.\footnote{
The finiteness condition can be relaxed a bit for signatures
that include certain commonly used sets of constants such as 
the set of all numerals.
}
By ``formally'' here 
we mean written in a machine-readable and processable format,
as opposed to written in free text, no matter how rigorously.
By this definition, theories themselves are defined informally, 
in natural language.
%
Some theories, such as the theory of bit vectors, have an infinite signature.
For them, the signature too is specified informally in English.\endnote{
\label{informal-signature}
To define such theory signatures formally,
SMT-LIB would need to rely on a more powerful underlying logic,
for instance one with dependent types.
}

%-------------------------------------------------------------------------------
\section{Input Formulas}
%-------------------------------------------------------------------------------

\new{SMT-LIB adopts a single and general (sorted) language for writing
logical formulas.}
It is often the case, however, that 
SMT applications work with formulas expressed in some particular fragment 
of the language.
The fragment in question matters
because one can often write a solver specialized on that sublanguage 
that is much more efficient than a solver meant for a larger sublanguage.\footnote{
By efficiency here we do not necessarily refer to worst-case time complexity,
but efficiency in practice.
}

An extreme case of this situation occurs
when satisfiability modulo a given theory $\T$
is decidable for a certain fragment (quantifier-free, say)
but undecidable for a larger one (full first-order, say),
as for instance happens with the theory of arrays~\cite{BraMS-VMCAI-06}.
But a similar situation occurs even
when the decidability of the satisfiability problem is preserved 
across various fragments.
For instance, 
if $\T$ is the theory of real numbers,
the satisfiability in $\T$ of full-first order formulas
built with the symbols $\{0,1,+,*,<,=\}$ is decidable.
However, one can implement increasingly faster solvers by restricting
the language respectively to quantifier-free formulas, 
linear equations and inequations,
difference inequations (inequations of the form $x < y + n$),
and
inequations between variables~\cite{Bozzanoetal2005TACAS}.

Certain pairs of theories and input languages are very common in the field
and are often conveniently considered as a single entity.
In recognition of this practice,
the SMT-LIB format allows one to pair together 
a background theory and an input language into a \emph{sublogic}, 
or, more briefly, \define{logic}.
We call these pairs (sub)logics because,
intuitively, 
each of them defines a sublogic of SMT-LIB logic
for restricting 
both the set of allowed models---to the models of the background theory---and 
the set of allowed formulas---to the formulas in the input language.


%-------------------------------------------------------------------------------
\section{Interface}
%-------------------------------------------------------------------------------

Starting with Version 2.0, the SMT-LIB standard includes a scripting language that 
defines a textual interface for SMT solvers.
SMT solvers implementing this interface act as interpreters
of the scripting language.
The language is command-based,
and defines a number of input/output functionalities
that go well beyond simply checking the satisfiability of an input formula.
It includes commands for 
setting various solver parameters,
declaring new symbols, 
asserting and retracting formulas, 
checking the satisfiability of the current set of asserted formulas, 
inquiring about models of satisfiable sets,
printing various diagnostics, and so on.

