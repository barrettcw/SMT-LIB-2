% commands.tex

%!TEX root = main.tex


%=================================================================================
\chapter{Operational Semantics of SMT-LIB} \label{chap:operational-semantics}
%=================================================================================
\thispagestyle{empty}


This chapter specifies how a human user or a software client can interact with 
an SMT-LIB-compliant solver. 
We do that by providing, as precisely as possible, an operational semantics 
of SMT-LIB scripts, together with additional requirements on the input/output 
behavior of the solver.

The expected interaction mode with a compliant solver is 
that of a read-eval-print loop: 
the user or client application issues a command in the format 
of the Command Language to the SMT solver via the solver's 
standard textual input channel;
the solver then responds over two textual output channels, 
one for regular output and one for diagnostic output,
and waits for another command.
A non-interactive mode is also allowed where the solver reads commands 
from a script stored in a file.
However, the solver's output behavior should be exactly the same as 
if the commands in the script had been sent to it one at a time.
 
Note that the primary goal of the SMT-LIB standard is, first and foremost,
to support convenient interaction with other programs, not human interaction.  
This has some influence on the design of the command language.  

There are other commands one might wish for an SMT solver to support
beyond those adopted here.  
In general, it is expected that time and more experience with the needs 
of applications will drive the addition of further commands in later versions.


%-------------------------------------------------------------------------------
\section{General Requirements} \label{sec:requirements}
%-------------------------------------------------------------------------------

The command language contains commands for managing a stack 
of \define{assertion levels} and for making queries about them.
It includes commands to:
%
\begin{itemize}
\item
declare and define new sort and function symbols
(\define{declare} and \define{define} commands),
\item
add formulas to the current assertion level,
\item
reset the assertion stack or the whole solver,
\item
push and pop assertion levels,
\item
check the joint satisfiability of all formulas in the assertion stack, 
possibly under additional assumptions (\define{check commands}),
\item
obtain further information following a check command 
(e.g., model information),
\item
set values for standard and solver-specific options, 
\item
get standard and solver-specific information from the solver.
\end{itemize}

This section provides some background and general requirements on how 
these functionalities are to be supported.
The next section, Section~\ref{sec:commands} provides more details
on how each compliant solver is to execute each command.

%-------------------------------------------------------------------------------
\subsection{Execution Modes} \label{sec:modes}
%-------------------------------------------------------------------------------

\begin{figure} %[t]
\begin{center}
\begin{tikzpicture}[->,>=stealth']
 \node[state,    	% layout (defined above)
  text width=2.5cm, 	% distance to ASSERT
  yshift=2cm, 		% move 2cm in y
  anchor=center] (START) 
 {\textbf{Start mode}};
  
 \node[state,    	% layout (defined above)
  text width=3cm,
%  yshift=2cm, 		% move 2cm in y
  right of=START, 	% Position is to the right of ASSERT
  node distance=4.5cm, 	% distance to ASSERT
  anchor=center] (ASSERT) 
 {\textbf{Assert mode}};
  
 % State: SAT with different content
 \node[state,    	% layout (defined above)
  text width=2cm, 	% max text width
  yshift=2cm, 		% move 2cm in y
  right of=ASSERT, 	% Position is to the right of ASSERT
  node distance=6.5cm, 	% distance to ASSERT
  anchor=center] (SAT) 	% posistion relative to the center of the 'box'
 {\textbf{Sat mode}};
 
 % STATE UNSAT
 \node[state,
  below of=SAT,
  yshift=-3cm,
  anchor=center,
  text width=2.5cm] (UNSAT) 
 {\textbf{Unsat mode}};

 % draw the paths and and print some Text below/above the graph
 \path 
 (START)	edge[loop above]	node[anchor=north,above]{\sf e, gsio, r, ra} (START)
 (START)	edge[bend left=10]	node[anchor=north,above]{\sf sl} (ASSERT)

 (ASSERT)	edge[loop above]	node[anchor=north,above]{\sf \begin{tabular}{c} ad, e, g, \\ gsio, p, ra\end{tabular}} (ASSERT)
 (ASSERT)	edge[bend right=16]	node[anchor=south,above]{\sf c} (SAT)
 (ASSERT)	edge[bend left=16]	node[anchor=north,above]{\sf c} (UNSAT)
 (ASSERT)	edge[bend left=10]	node[anchor=south,below]{\sf r} (START)

 (SAT)		edge[loop above]	node[anchor=south,above]{\sf c, e, g, gamv, gsio} (SAT)
 (SAT)		edge[bend right=24]	node[anchor=south,above]{\sf r} (START)
 (SAT)		edge[bend right=20]	node[anchor=south,above]{\sf ad, p, ra} (ASSERT)
 (SAT)		edge[bend left=16]	node[anchor=south,right]{\sf c} (UNSAT)

 (UNSAT)	edge[loop below]	node[anchor=north,below]{\sf c, e, g, gsio, gpu} (UNSAT)
 (UNSAT)	edge[bend left=24]	node[anchor=south,above]{\sf r} (START)
 (UNSAT)	edge[bend left=20]	node[anchor=south,above]{\sf ad, p, ra} (ASSERT)
 (UNSAT)	edge[bend left=16]	node[anchor=south,left]{\sf c} (SAT)
 ;
\end{tikzpicture}
\end{center}
%
Command name abbreviations:
{\scriptsize
\medskip

\begin{multicols}{2}
\begin{tabular}{r@{\ \ }c@{\ \ }l}
{\sf ad} & = & \ter{assert}, \ter{declare-}{*}, \ter{define-}{*}
\\ 
{\sf c} & = & \ter{check-sat}{*} 
\\ 
{\sf e} & = & \ter{echo} \\ 
{\sf g} & = & \ter{get-assertions} \\ 
{\sf gamv} & = & \ter{get-assignment}, \ter{get-model}, \ter{get-value} \\
{\sf gsio} & = & \ter{get-info}, \ter{get-option}, \ter{set-info}, \ter{set-option} \\  
\end{tabular}

\begin{tabular}{r@{\ \ }c@{\ \ }l}
{\sf gpu} & = & \ter{get-proof}, \ter{get-unsat-}{*} \\ 
{\sf p} & = & \ter{pop}, \ter{push} \\ 
{\sf r} & = & \ter{reset} \\ 
{\sf ra} & = & \ter{reset-assertions} \\ 
{\sf sl} & = & \ter{set-logic}
\end{tabular}
\end{multicols}
}
\caption{Abstract view of transitions between solver execution modes.
The symbol * here stands for the matching wildcard.
}

\label{fig:modes}
\end{figure}


At a high-level, a compliant solver can be understood as being at all times 
in one of four \define{execution modes}:
a \mode{start} mode, an \mode{assert} mode and two \define{query} modes, 
\mode{sat} and \mode{unsat}.
The solver starts in \mode{start} mode, moves to \mode{assert mode} 
once a logic is set, and then moves to one of the two query modes 
after executing a check command.
Any command other than \ter{reset} that modifies the assertion stack brings 
the solver back from a query mode to the \mode{assert} mode.
The \ter{reset} command takes the solver back to \mode{start} mode.

The transition system in Figure~\ref{fig:modes} illustrates 
in some detail which commands can trigger which mode transitions.
The set of labels for each transition describes the commands that may cause it.
With the exception of \ter{exit}, 
if a command does not appear on any transitions originating from a mode,
it is not permitted in that mode.
The \ter{exit} command, which causes the solver to quit, can be issued in any mode.

The solver must respond with an error when given a command not permitted 
in the current mode.
Because of its level of abstraction, the transition system diagram 
in Figure~\ref{fig:modes} does not specify the conditions under which 
a specific command causes a transition to one mode as opposed to another;
see Section~\ref{sec:commands} for details on that.
Similarly, the diagram does not account for the fact that some specific options
can be set only in certain modes. 
Such restrictions are described in Section~\ref{sec:options}. 

%-------------------------------------------------------------------------------
\subsection{Solver responses} \label{sec:response}
%-------------------------------------------------------------------------------

Regular output, including responses \emph{and errors}, produced 
by compliant solvers should be written to the \define{regular} output channel.
Diagnostic output, including warnings, debugging, tracing, 
or progress information, should be written to the \define{diagnostic} output channel.
These channels may be set with the \nter{set-option} command 
(see Section~\ref{sec:options} below).  
By default they are the standard output and standard error channels, 
respectively.

Generally, once a solver completes its processing in response to a command, 
it should print to its regular output channel a \nter{general\_response}:

\begin{center}
 \begin{tabular}{lll}
  \cGeneralResponse
 \end{tabular}
\end{center}

\noindent 
The value \ter{success} is the default response for a successful execution
of a supported command.  
A number of commands have a more detailed response in place of \ter{success}, 
discussed in Section~\ref{sec:commands} for each of them.
The value \ter{unsupported} should be returned if the command 
or some specific input to it is not supported by the solver.
An expression of the form \expr{(error $e$)} should be returned for any 
kind of error situation (wrong command syntax, incorrect parameters, 
erroneous execution, and so on).
The value of $e$ is a solver-specific string containing 
a message that describes the problem encountered.\footnote{%
Returning the empty string is allowed but discouraged 
because of its uninformative content.
}

Any response which is not double-quoted and not parenthesized must be followed 
by at least one whitespace character (for example, a new line character).\endnote{
This enables applications reading a compliant solver's response to know 
when an identifier (like \ter{success}) has been completely printed and,
in general, when the solver has completed processing a command. 
For example, this is needed if one wants to use an off-the-shelf 
S-expression parser (e.g., \texttt{read} in Common Lisp) to read responses.
}

Several options described in Section~\ref{sec:options} below affect the printing 
of responses, in particular by suppressing the printing of \ter{success}, 
or by redirecting the regular or diagnostic output channels.

\paragraph{Errors and solver state.} 
Solvers have two options when encountering errors.  
For both options, they first print an error message 
in the \nter{general\_response} format.  
Then, they may either immediately exit with a non-zero exit status, 
or continue accepting commands.  
In the second case, 
the solver's state remains unmodified by the error-generating command, 
except possibly for timing and diagnostic information.  
In particular, the assertion stack, discussed in Section~\ref{sec:assert-stack}, 
is unchanged.\endnote{
The motivation for allowing these two behaviors is that
the first one (exiting immediately when an error occurs) may be simpler to implement,
while the latter may be more useful for applications, though it might be 
more burdensome to support the semantics of leaving the state unmodified 
by the erroneous command.
}

The predefined \attr{error-behavior} attribute can be used 
with the $\ter{get-info}$ command to check which error behavior 
the tool supports (see Section~\ref{sec:get-info} below).

%-------------------------------------------------------------------------------
\subsection{Printing of terms and defined symbols} 
%-------------------------------------------------------------------------------
Several commands request the solver to print sets of terms.  
While some commands, naturally, place additional semantic requirements on these sets, 
the general syntactic requirement is that output terms must be well-sorted 
with respect to the current signature (as defined below in Section~\ref{sec:assert-stack}).
%
%ct remove this because get-rank is out for the time being
%If the signature is expanded, due to
%the introduction of new symbols during solving, then the ranks of
%those new symbols can be obtained with $\akey{get-rank}$.

All output from a compliant solver should print any symbols
defined with \ter{define-sort} and \ter{define-fun} just as they are, 
without replacing them by the expression they are defined to be equal to.  
This approach generally keeps the output from solvers much more compact than 
it would be if definitions were expanded.

%ct removed for now, we need to agree on the format for this
%
%\paragraph{Getting a symbol's rank.} The $\akey{get-rank}$ command allows
%the calling environment to request the sort of a symbol.  In case the
%symbol is overloaded, all ranks should be returned.  This command is
%likely to be particularly useful to determine the sort of
%solver-introduced symbols.



%-------------------------------------------------------------------------------
\subsection{The assertion stack} \label{sec:assert-stack}
%-------------------------------------------------------------------------------

A compliant solver maintains a stack of sets, 
each of which consists of \define{assertions}.
Assertions are formulas, declarations, and definitions.
We will use the following terminology with regards to this data structure:
\begin{itemize}
\item 
\define{assertion stack:} the single stack of sets of assertions;
\item 
\define{assertion level:} an element of the assertion stack 
(i.e., a set of assertions);
\item 
\define{context:} the union of all the assertion levels on the assertion stack
together with any global declarations (see Section~\ref{sec:decl-defn});
\item 
\define{current assertion level:} the assertion level at the top of the stack 
(i.e., the most recent);
\item \define{first assertion level:} the first assertion level in the stack
(i.e., the least recent);
\item \define{current signature:} the signature determined by the logic
  specified with the most recent \ter{set-logic} command and
by the set of sort symbols and rank associations (for function symbols) 
in the current context.  
\end{itemize}

Initially, when the solver starts, the assertion stack consists 
of a single element, the first assertion level, which is empty.
While new assertions can be added to this set, the set itself cannot be removed 
from the stack with a pop operation.
The following commands modify the current context:
\begin{quote}
\ter{assert},
\ter{declare-sort}, \ter{declare-fun},
\ter{declare-const},
\ter{define-sort}, \ter{define-fun}, 
\ter{define-fun-rec}, \ter{define-funs-rec},
\ter{pop}, \ter{push},
\ter{reset}, and \ter{reset-assertions}.
\end{quote}
%\footnote{%
%A \ter{check-sat-assuming} command modifies the context by temporarily asserting
%some formulas as \define{assumptions} (see Section~\ref{sec:checking-for-sat}).
%}

%-------------------------------------------------------------------------------
\subsection{Symbol declarations and definitions}\label{sec:decl-defn}
%-------------------------------------------------------------------------------

A number of commands allow the declaration or definition of a function 
or sort symbol.  
By default, these declarations and definitions are added to the current
assertion level when the corresponding command is executed.
Popping that assertion level removes them.\endnote{
It is desirable to have the ability to remove declarations and definitions, 
for example if they are no longer needed at some point during an interaction 
with a solver (so that the memory required for them can be reclaimed), or 
if a defined symbol needs to be redefined.  
The current approach of allowing declarations and definitions to be 
locally scoped supports removal by popping the containing assertion level.
Other approaches, such as the ability to add shadowing declarations or
definitions of symbols, or to ``undefine'' or ``undeclare'' them, 
present some issues: for example, how to print symbols that have been shadowed, 
undefined or undeclared.
As a consequence, they are not supported in the language.
}
As an alternative, declarations and definitions can all be made \define{global} 
by running the solver with the option \attr{global-declarations} set to \ter{true}.
When running with this option set, all declarations and definitions become permanent.
That is, they survive any pop operations on the assertion stack as well as invocations
of \ter{reset-assertions} and can only be removed by a global reset, 
achieved with the \ter{reset} command.\endnote{
Setting \attr{global-declarations} to \ter{true} can be understood as stating 
that declarations and definitions are not part of the assertion stack, 
and so resetting the stack has no impact on them. 
This option is convenient in applications that benefit from using push and pop
for assertions but need to continue using symbols declared after a push 
even after the corresponding pop.
}

Well-sortedness checks, required for commands that use sorts or terms, 
are always done with respect to the current signature.
It is an error to declare or define a symbol
that is already in the current signature.
This implies in particular that, contrary to theory function symbols,
user-defined function symbols cannot be overloaded.\endnote{
The motivation for not overloading user-defined symbols is to simplify 
their processing by a solver.
This restriction is significant only for users 
who want to extend the signature of the theory used by a script 
with a new polymorphic function symbol---i.e., one whose rank would contain
parametric sorts if it was a theory symbol.
For instance, users who want to declare a ``reverse'' function 
on arbitrary lists, must define a different reverse function symbol 
for each (concrete) list sort used in the script.
This restriction might be removed in future versions.
}


%-------------------------------------------------------------------------------
\subsection{In-line definitions}\label{sec:inline-defn}
%-------------------------------------------------------------------------------

Any closed subterm $t$ occurring in the argument(s) of a command $c$
can be optionally annotated with a \attr{named} attribute; that is, it
can appear as \expr{(!~$t$ \attr{named} $f$)} where $f$ is a fresh function symbol
from \nter{symbol}.
For such a command $c$, let
\begin{center}
\expr{(!~$t_1$ \attr{named} $f_1$)}, \ldots, \expr{(!~$t_n$ \attr{named} $f_n$)}
\end{center}
%
be the enumeration of all the named subterms of $c$ obtained by the (depth-first) left-to-right post-order traversal of $c$.
The semantics of the command $c$ is the same as the sequence of commands
\begin{center}
\begin{tabular}{l}
 \expr{(define-fun $f_1$ () $\sigma_1$ $t_1'$)} \\
 \vdots \\
 \expr{(define-fun $f_n$ () $\sigma_n$ $t_n'$)} \\
 $c'$
\end{tabular}
\end{center}
%
where, for each $i=1,\ldots,n$, 
$(i)$ $\sigma_i$ is the sort of $t_i$ with respect to the current signature 
up to the declaration of $f_i$,
$(ii)$ $t_i'$ is the term obtained from $t_i$ by removing 
all its \attr{named} annotations, and 
$(iii)$ $c'$ is similarly obtained from $c$ by removing 
all its \attr{named} annotations.

By these semantics, each \define{label} $f_i$ can occur, as a constant symbol, 
in any subexpression of $c$ that comes after \expr{(!~$t_i$ \attr{named} $f_i$)} 
in the left-to-right post-order traversal of $c$, as well as after the command $c$ itself.
The labels $f_1, \ldots, f_n$ can be used like any other user-defined nullary
function symbols, with the same visibility and scoping restrictions 
they would have if they had been defined with the sequence of commands above. 
However, contrary to function symbols introduced by \expr{define-fun},
labels have an additional, dedicated use in the commands 
\expr{get-assignment} and \expr{get-unsat-core}
(see Section~\ref{sec:commands}).


%------------------------------------------------------------------------
\subsection{Solver options} \label{sec:options}
%------------------------------------------------------------------------

Solver options may be set using the \ter{set-option} command, and
their current values can be obtained using the \ter{get-option} command.  
If a solver does not support the setting of a particular option,
for either command it should output \ter{unsupported}.  

Solver-specific option names are allowed and indeed expected.  
A set of standard options is presented in this subsection;
refer to Figure~\ref{fig:command-options} for their format. 
%This version of the language requires solvers to recognize and reply 
%in a standard way only to a few of them, the rest are optional.
We discuss each option below, specifying also their default values and 
whether or not compliant solvers are required to support them. 
It is understood that if a solver does not support one of the optional 
standard options below, it behaves as if that option was permanently
set to its default value. 

Some options can be set only when the solver is in \mode{start} mode.
We list the mode when that is the case.
Attempting to set an option when the solver is not in a permitted mode 
should trigger an error response.
Each option starting with the \ter{produce-} prefix is a Boolean option
that enables a specific command.
When such an option is set to \ter{false}, calling the corresponding
command should trigger an error response.

The set of standard options is likely to be expanded or otherwise revised 
as further desirable common options and kinds of information across tools
are identified.  

\begin{description}

\item[\attr{diagnostic-output-channel}] 
\quad default: \ter{"stderr"}
\quad support: required
\\
The argument is a string consisting of the name of a file to be used subsequently 
as the diagnostic output channel.
The input value \ter{"stderr"} is interpreted specially to mean 
the solver's standard error channel.  
With other filenames, subsequent solver output is to be appended to the named file 
(and the file should be first created if it does not  already exist).

\item[\attr{global-declarations}] 
\quad default: \ter{false}
\quad support: optional
\quad mode: \mode{start}
\\
If the solver supports this option, setting it to \ter{true} causes all
declarations and definitions to be global (permanent) as opposed to being added
to the current assertion level.

\item[\attr{interactive-mode}] 
\quad default: \ter{false}
\quad support: optional
\quad mode: \mode{start}
\\
The old name for \ter{produce-assertions}. 
Deprecated.

\item[\attr{print-success}] 
\quad default: \ter{false}
\quad support: required
\\
Setting this option to \ter{true} causes the solver to print 
\ter{success} as a response to commands.  
Other output remains unchanged.

\item[\attr{produce-assertions}] 
\quad default: \ter{false}
\quad support: optional
\quad mode: \mode{start}
\\
If the solver supports this option, setting it to \ter{true} enables 
the \ter{get-assertions} command.
This option was called \ter{interactive-mode} in previous versions.

\item[\attr{produce-assignments}] 
\quad default: \ter{false}
\quad support: optional
\quad mode: \mode{start}
\\
If supported, this enables the command \ter{get-assignment}.  

\item[\attr{produce-models}] 
\quad default: \ter{false}
\quad support: optional
\quad mode: \mode{start}
\\
If supported, this enables the commands \ter{get-value} and \ter{get-model}.  

\item[\attr{produce-proofs}] 
\quad default: \ter{false}
\quad support: optional
\quad mode: \mode{start}
\\
If supported, this enables the command \ter{get-proof}.  

\item[\attr{produce-unsat-assumptions}] 
\quad default: \ter{false}
\quad support: optional
\quad mode: \mode{start}
\\
If supported, this enables the command \ter{get-unsat-assumptions}.  

\item[\attr{produce-unsat-cores}] 
\quad default: \ter{false}
\quad support: optional
\quad mode: \mode{start}
\\
If supported, this enables the command \ter{get-unsat-core}.  

\item[\attr{random-seed}] 
\quad default: $0$
\quad support: optional
\quad mode: \mode{start}
\\
The argument is a numeral for the solver to use as a random seed,
in case the solver uses (pseudo-)randomization.
The default value of $0$ means that the solver can use any random seed---possibly 
even a different one for each run of the same script.
The intended use of the option is to force the solver to produce identical results 
whenever given identical input (including identical non-zero seeds) on repeated runs of the solver.

\item[\attr{regular-output-channel}] 
\quad default: \ter{"stdout"}
\quad support: required
\\
The argument should be a filename to use subsequently for
the regular output channel.  The input value \ter{"stdout"} is interpreted
specially to mean the solver's standard output channel.
With other filenames, subsequent solver output is to be appended to the named file 
(and the file should be first created if it does not  already exist).

\item[\attr{reproducible-resource-limit}] 
\quad default: $0$
\quad support: optional
\\
If the solver supports this option, setting it to $0$ disables it.
Setting it a non-zero numeral $n$ will cause each subsequent check command 
to terminate within a bounded amount of time dependent on $n$.
The internal implementation of this option and its relation to run time or 
other concrete resources can be solver-specific.
However, it is required that the invocation of a check command return 
\ter{unknown} whenever the solver is unable to determine the satisfiability 
of the formulas in the current context within the current resource limit. 
Setting a higher value of $n$ should allow more resources to be used, 
which may cause the command to return \ter{sat} or \ter{unsat} instead 
of \ter{unknown}. 
Furthermore, the returned result should depend deterministically on $n$;
specifically, it should be the same every time the solver is run 
with the same sequence of previous commands on the same machine 
(and with an arbitrarily long external time out). 
If the solver makes use of randomization, it may require 
the \attr{random-seed} option to be set to a value other than $0$ 
before \attr{reproducible-resource-limit} can be set to a positive value.\endnote{
Note that this option is not intended to be used for comparisons
between different solvers since they can implement it differently.
Its purpose is simply to guarantee the reproducibility 
of an individual solver's results under the same external conditions.
}

\item[\attr{verbosity}] 
\quad default: no standard default value
\quad support: optional
\\
The argument is a numeral controlling the level of diagnostic output produced
by the solver.  
All such output should be written to the diagnostic output channel\endnote{
This is to avoid confusion with the responses to commands,
which are written to the regular output channel.  
}
which can be set and later changed via the \ter{diagnostic-output-channel} option.  
An argument of $0$ requests that no such output be produced.  
Higher values correspond to more verbose output.
\end{description}

%---------------------------------------------------------------------------
\subsection{Solver information}   \label{sec:get-info}
%---------------------------------------------------------------------------

The format for responses to \ter{get-info} commands,
both for standard and solver-specific information flags,
is defined by the \nter{get\_info\_response} category 
in Figure~\ref{fig:command-responses}.
The standard \ter{get-info} flags and specific formats 
for their corresponding responses are given next.  

\begin{description}

\item[\attr{all-statistics}] 
\quad support: optional
\quad mode: \mode{sat}, \mode{unsat}
\\
Solvers reply with a parenthesis-delimited sequence of \nter{info\_response} 
values (see Figure~\ref{fig:command-responses}) providing various statistics 
on the execution of the most recent check command.
No standard statistics are defined for the time being,\endnote{
Some commonly used statistics 
(e.g., number of restarts of a solver's propositional reasoning engine)
are difficult to define precisely and generally, 
while the exact semantics of others, such as time and memory usage,
have not being agreed upon yet by the SMT community.
}
so they are all solver-specific.
%ct reworded
%Any call to \ter{get-info} with \attr{all-statistics} must follow 
%a check command and precede any commands that modify the assertion stack.  
Executions of \ter{get-info} with \attr{all-statistics} are allowed only
when the solver is in \mode{sat} or \mode{unsat} mode.

\item[\attr{assertion-stack-levels}]
\quad support: optional
\\
The response is a pair of the form
\expr{(\attr{assertion-stack-levels} $n$)}
where $n$ is a numeral indicating the current number of levels 
in the assertion stack besides the first assertion level.\endnote{
This command is useful for interactive use,
to keep track of the current number of nested \ter{push} commands.
}

\item[\attr{authors}] 
\quad support: required
\\
The response is a pair of the form
\expr{(\attr{authors} $s$)}
where $s$ is a string literal listing the names of the solver's authors.

\item[\attr{error-behavior}]
\quad support: required
\\  
The response is a pair of the form
\expr{(\attr{error-behavior} $r$)}
where $r$ is either \ter{immediate-exit} or \ter{continued-execution}.
A response of \ter{immediate-exit} indicates that the solver
will exit immediately when an error is encountered.  
A response of \ter{continued-execution} indicates that when an error is
encountered, the solver
will return to the state it was in immediately before the command triggering
the error, and continue 
accepting and executing new commands.  
See Section~\ref{sec:response} for more information.

\item[\attr{name}]
\quad support: required
\\
The response is a pair of the form
\expr{(\attr{name} $s$)}
where $s$ is a string literal with the name of the solver.

\item[\attr{reason-unknown}]
\quad support: optional
\quad mode: \mode{sat}
\\
%ct reworded
%Any call to \ter{get-info} with \attr{reason-unknown} must follow a check command
%whose response was \ter{unknown}.  In addition, there must be no intervening commands 
%that modify the assertion stack.
Executions of \ter{get-info} with \attr{reason-unknown} are allowed only
when the solver is in \mode{sat} mode following a check command
whose response was \ter{unknown}.
%
The response is a pair of the form
\expr{(\attr{reason-unknown} $r$)}
where $r$ is an element of \nter{reason\_unknown} giving a short reason 
why the solver could not successfully check satisfiability.
In general, this reason can be provided by a solver-defined s-expression.
Two predefined s-expressions are
\ter{memout}, for out of memory, and \ter{incomplete}, which indicates 
that the solver knows it is incomplete for the class of formulas containing 
the most recent check query.

\item[\attr{version}]
\quad support: required
\\ 
The response is a string literal with the version number of the solver 
(e.g., \texttt{"1.2"}).
\end{description}


%-------------------------------------------------------------------------------
\section{Commands} \label{sec:commands}
%-------------------------------------------------------------------------------

The full set of commands and their expected behavior are described 
in this section.
Commands may impose restrictions on their arguments as well as restrictions 
on when they can be issued.
Unless otherwise specified, the solver is required to produce an error 
when any of these restrictions is violated.
Figure~\ref{fig:modes} describes the commands permitted in each execution mode 
and the mode transitions each command may trigger.
We specify below the conditions under which a command triggers one mode transition 
versus another only for commands that may trigger more than one transition. 

%-------------------------------------------------------------------------------
\subsection{(Re)starting and terminating} \label{sec:restarting}
%-------------------------------------------------------------------------------

\begin{description}

%%% reset %%%
%
\item[\expr{(reset)}]
resets the solver completely to the state it had after it was started
and before it started reading commands.\endnote{
This allows the user to reset the state of the solver 
without paying the cost of restarting it.
}
\smallskip

%%% set-logic %%%
%
\item[\expr{(set-logic $l$)}]
tells the solver what logic, in the sense of Section~\ref{sec:logics},
is being used.
The argument $l$ can be the name of a logic in the SMT-LIB catalog or 
of some other solver-specific logic.
The effect of the command is to add globally (and permanently)
a declaration of each sort and function symbol in the logic.

The argument $l$ can also be the predefined symbol \ter{ALL}.
With this argument, the solver sets the logic to the most general logic 
it supports.\endnote{
Having \ter{ALL} is convenient for client applications that generate 
problems on the fly in a variety of logics supported by some specific solver
without knowing in advance the specific logic for each problem.
SMT-LIB scripts meant for the SMT-LIB library should not use \ter{ALL} 
in set-logic but should instead use a specific logic name.
}
Note that while the reaction to \expr{(set-logic ALL)} is the same 
for every compliant solver, the chosen logic is solver-specific.

We refer to the logic set by the most recent \ter{set-logic} command as 
the \define{current logic}.
\smallskip

%%% set-option %%%
%
\item[\expr{(set-option $o$ $v$)}]
sets a solver's option $o$ to a specific value $v$.
More details on predefined options and required behavior are provided 
in Section~\ref{sec:options}.  
In general, if a solver does not support the setting of a particular option,
its response to this command should be \ter{unsupported}.
If the option is one of the predefined ones it should also leave it unchanged 
from its default value.  
The effect of setting a supported option is immediate.
In particular, for options that affect the solver's output, such as 
\attr{diagnostic-output-channel}, \attr{regular-output-channel} and 
\attr{print-success}, 
the effect applies already to the output of the very command 
that is setting the option.

Note that some of the options defined in Section~\ref{sec:options} 
may only be set in \mode{start} mode.\endnote{
The rationale is that a solver may need to make substantial changes 
to its internal configuration to provide the functionality requested 
by these options, and so it needs to be notified before it starts 
processing assertions.
}
\smallskip

%%% exit %%%
%
\item[\expr{(exit)}]
instructs the solver to exit.
\end{description}


%-------------------------------------------------------------------------------
\subsection{Modifying the assertion stack}
%-------------------------------------------------------------------------------

\begin{description}
%%% push %%%
%
\item[\expr{(push $n$)}] 
pushes $n$ empty assertion levels onto the assertion stack.\footnote{%
Typically, $n=1$.
}
If $n$ is $0$, no assertion levels are pushed.
\smallskip

%%% pop %%%
%
\item[\expr{(pop $n$)}] where $n$ is smaller than the number of assertion levels
in the stack, pops the $n$ most-recent assertion levels from the stack.\footnote{%
When $n$ is $0$, no assertion levels are popped.
}
Note that the first assertion level, 
which is not created by a \ter{push} command, 
cannot be popped.
\smallskip

%%% reset-assertions %%%
%
\item[\expr{(reset-assertions)}] removes from the assertion stack 
all assertion levels beyond the first one.
In addition, it removes all assertions from the first assertion level.
Declarations and definitions resulting from the \ter{set-logic} command 
are unaffected (because they are global).
Similarly, if the option \attr{global-declarations} has value \ter{true}
at the time the command is executed, then \emph{all} declarations and definitions
remain unaffected.
Note that any information set with \ter{set-option} commands is preserved 
in any case.

\end{description}


%-------------------------------------------------------------------------------
\subsection{Introducing new symbols} \label{sec:new-symbols}
%-------------------------------------------------------------------------------

\new{The first command below allows the declaration of sort parameters.}
The next seven commands allow one to introduce new sort or function symbols 
by providing them with a rank declaration
(\ter{declare-sort}, \ter{declare-fun} and \ter{declare-const})
or also with a definition 
(\ter{define-sort}, \ter{define-fun}, \ter{define-fun-rec} and 
\ter{define-funs-rec}).
We refer to the former as \define{user-declared} symbols and 
the latter as \define{user-defined} symbols.
Declarations and definitions are made global (permanent) or are added 
to the current assertion level depending on 
whether the option \attr{global-declarations} is set to \ter{true} or not.

\begin{description}

%%% declare-sort-parameter %%%
%
\item[\expr{(declare-sort-parameter $s$)}]
\new{
adds \emph{global} sort parameter $s$ to the current signature.
The command reports an error if $s$ is a sort symbol, sort parameter, 
or theory symbol already present in the current signature.%
}
%\endnote{%
%\new{
%In SMT-LIB 2.7, the \expr{par} construct, short for \emph{parameter}, is a misnomer,
%as \expr{par} actually introduces sort \emph{variables} rather than sort \emph{parameters}.
%We have adopted a pragmatic approach to this issue, 
%deciding not to rename the \expr{par} construct for backward compatibility.
%}
%}
\smallskip

%%% declare-sort %%%
%
\item[\expr{(declare-sort $s$ $n$)}]
adds sort symbol $s$ with associated arity $n$.
It is an error if $s$ is a sort symbol \new{or parameter} already present 
in the current signature.
\smallskip

%%% define-sort %%%
%
\item[\expr{(define-sort $s$ ($u_1$ $\cdots$ $u_n$) $\tau$)}] with $n \geq 0$
adds sort symbol $s$ with arity $n$ and associates it with sort $\tau$,
where the $u_i$'s are (local) sort parameters, and $\tau$ may contain 
any of $u_1$, \dots, $u_n$\new{, but no global sort parameters}.

% (define-sort set-of-pairs (u v) (set (pair (u v))))
% (define-sort set-of-pairs () (set (pair (u v))))
% (set-of-pairs Int Real)

Subsequent well-sortedness checks must treat a sort term like
\expr{($s$ $\tau_1$ $\cdots$ $\tau_n$)} 
as an abbreviation for the term obtained by simultaneously substituting
$\tau_i$ for $u_i$, for $i\in\{1,\ldots,n\}$, in $\tau$.\endnote{
Strictly speaking, only sort symbols introduced with \ter{declare-sort}
expand the initial signature of theory sort symbols.
Sort symbols introduced with \ter{define-sort} do not.
They do not construct \emph{real} sorts,
but \emph{aliases} of sorts built with
theory sort symbols and previously declared sort symbols.
}

The command reports an error
if $s$ is a sort symbol or parameter already present in the current signature or
if $\tau$ is not a well-defined (polymorphic) sort 
with respect to the current signature extended with sort parameters 
$u_1$, \dots, $u_n$.
This restriction prohibits (meaningless) circular definitions
where $\tau$ contains $s$.
\begin{newver}
The $u_i$ sort parameters shadow any previously user-declared sort symbol
or parameter with the same name.
However, they do not shadow theory sort symbols (just as bound variables are
not permitted to shadow declared theory symbols).
The command reports an error if a theory sort
symbol is used as a sort parameter $u_i$.
\end{newver}

\smallskip

%%% declare-fun %%%
%
\item[\expr{(declare-fun $f$ ($\tau_1$ $\cdots$ $\tau_n$) $\tau$)}]
with $n \geq 0$ adds a new symbol $f$ 
with associated rank $\tau_1\cdots\tau_n\tau$.
The command reports an error if a function symbol with name $f$ 
is already present in the current signature.
\new{
Note that 
$\tau_1$ $\cdots$ $\tau_n$, and $\tau$ may contain sort parameters,
making $f$ polymorphic.
}
\smallskip

%%% declare-const %%%
%
\item[\expr{(declare-const $f$ $\tau$)}]
has the same effect as the command \expr{(declare-fun $f$ () $\tau$)}.
\smallskip

%\pfrem{sort parameters in datatypes must come from the par list.}

%\pfrem{using declared sort parameters only make sense in assertions.
%  We need order elsewhere.}
%%% declare-datatypes %%%
%
\item[%
\expr{(declare-datatypes (($\delta_1$ $k_1$) $\cdots$ ($\delta_n$ $k_n$))
                         ($d_1$ $\cdots$ $d_n$))}]
with $n > 0$ introduces $n$ algebraic datatypes $\delta_1,\dots,\delta_n$
with respective arities $k_1,\dots,k_n$ and declarations $d_1,\dots,d_n$. 
Let $\delta = \delta_i$, $k = k_i$ and $d = d_i$ for $i \in \{1,\ldots,n\}$.
If $k >0$ then $d$ is an expression of the form 
\expr{(par ($u_1$ $\cdots$ $u_k$) $l$)}
where $u_1, \ldots, u_k$ are sort parameters;\endnote{
\new{
Using local sort parameters as opposed to globally declared sort parameters
is necessary because their order is meaningful, as arguments of the sort constructor.
}
}
% The u_1, u_k correspond to the k arguments of the sort constructor.  
% (Pair Int Real) is not the same of (Pair Real Int), and the relation to the pair
% constructor and the order of arguments in the pair constructor is set through
% the order of the sort parameters in par
otherwise, it is just $l$. 
In either case, $l$ is a (parenthesis-delimited) list of \emph{one or more} 
expressions of the form
%
\begin{center}
 \expr{($c$ ($s_1$ $\tau_1$) $\cdots$ ($s_m$ $\tau_m$))},
\end{center}
where 
$(i)$ $m \geq 0$, 
$(ii)$ $c$ is a symbol, a \define{constructor for $\delta$}, and 
$(iii)$ for each $j=1,\ldots,m$, $s_j$ is a symbol, a \define{selector for $c$}, and 
$(iv)$ $\tau_j$ is a sort term
that contains no occurrences of $\delta_1, \ldots, \delta_n$ below its top symbol.

In the polymorphic case,
\new{the terms $\tau_i$ can contain only the sort parameters in the list $u_1, \ldots, u_k$}.%
\footnote{%
\new{
No previously declared sort parameter that is not shadowed by a $u_i$ can be used.
}
}
The datatype $\delta$ must be \define{well founded} in the following inductive sense:
it must have a constructor of rank $\tau_1\cdots\tau_m\delta$ 
such that $\tau_1\cdots\tau_m$ does not contain any of the datatypes 
from $\{\delta_1, \ldots, \delta_n\}$ or, if it does contain some, they are well founded.

A compliant solver must return an error in response to invocations of this command 
that do not satisfy all of the restrictions above.  

In the polymorphic case, 
the command has the effect of declaring each $\delta$ as a sort symbol of arity $k$;
each constructor $c$ as a function symbol of parametric rank $\tau_1\cdots\tau_m\delta$;
and each $s_i$ as a function symbol of parametric rank $\delta\tau_i$.
The monomorphic case is analogous.

Note that the sort terms $\tau_1, \ldots,\tau_m$ can contain any previously defined 
sort symbol as well as any of the datatypes $\delta_1, \ldots, \delta_n$,
as long as those datatypes are well founded.\endnote{
The input of this command is split in two arguments precisely to facilitate
the type-checking of the sort terms $\tau_1, \ldots,\tau_m$.
The first argument makes sure that when it is time to parse these terms
each symbol $\delta_i$ is known to be a sort symbol of arity $k_i$.
}
This allows the declaration of recursive and mutually recursive datatypes.\endnote{
The various restrictions on the definition of a datatype $\delta$ are crucial 
since they allow the existence of standard interpretations for $\delta$.
See Chapter~\ref{chap:logical-semantics} for more details.
}

On successfully executing this command, 
for each constructor $c$ in a declared datatype $\delta$, 
the solver will also automatically declare a \define{tester}
with rank $\delta\,\ter{Bool}$.
The tester's name is an indexed identifier (see Section~\ref{sec:identifiers}) 
of the form \expr{(\_ is $c$)}.
\medskip

\begin{lstlisting}[linewidth=37em]
; an enumeration datatype
(declare-datatypes ( (Color 0) ) (
  ( (red) (green) (blue) ))
)
; testers: (_ is red), (_ is green)

; integer lists with "empty" and "insert" constructors
(declare-datatypes ( (IntList 0) ) (
  ( (empty) (insert (head Int) (tail IntList) )))
)
; testers: (_ is empty), (_ is insert)

; parametric lists with "nil" and "cons" constructors
(declare-datatypes ( (List 1) ) (
  (par (T) ( (nil) (cons (car T) (cdr (List T)) )))))

; option datatype
(declare-datatypes ( (Option 1) ) (
  (par (X) ( (none) (some (val X)) ))))

; parametric pairs
(declare-datatypes ( (Pair 2) ) (
  (par (X Y) ( (pair (first X) (second Y)) ))))
  
; two mutually recursive datatypes
(declare-datatypes ( (Tree 1) (TreeList 1) ) (
  ; Tree
  (par (X) ( (node (value X) (children (TreeList X)) )))
  ; TreeList
  (par (Y) ( (empty) 
    (insert (head (Tree Y)) (tail (TreeList Y))) ))))
    
\end{lstlisting}

Since $\delta_1, \ldots, \delta_n$ are sort symbols, 
none of them can be a previously declared sort symbol.
Similarly, constructors and selectors are function symbols,
so none of them can be a previous declared/defined function symbol.
This has the effect of also prohibiting, for instance, the use 
of the same constructor in different datatypes or the use 
of repeated instances of the same selector in the same datatype.\endnote{
The rationale for this restriction is the same as for function symbols
introduced with \ter{declare-fun} and \ter{define-fun}, i.e., 
keep parsing and type checking simple. 
}
\smallskip

%%% declare-datatype %%%
%
\item[\expr{ 
(declare-datatype $\delta$ $d$)}] is an abbreviation of 
\begin{center}
\expr{(declare-datatypes (($\delta$ $0$)) ($d$))} 
\end{center}
if $\delta$ is not parametric, and an abbreviation of 
\begin{center}
\expr{(declare-datatypes (($\delta$ $k$)) ($d$))} 
\end{center}
if $d$ has the form \expr{(par ($u_1$ $\cdots$ $u_k$) $l$)}.
This command provides a simpler syntax for defining a single datatype.

\begin{lstlisting}[linewidth=37em]
; an enumeration datatype
(declare-datatype Color ( (red) (green) (blue) ))

(declare-datatype IntList 
  ( (empty) 
    (insert (head Int) (tail IntList) )))

(declare-datatype List (par (E)
  ( (nil) 
    (cons (car E) (cdr (List E)) ))))

(declare-datatype Option (par (X) 
  ( (none) 
    (some (val X) ))))

(declare-datatype Pair (par (X Y)
 ( (pair (first X) (second Y)) ))))
\end{lstlisting}
\smallskip

%%% define-fun %%%
%
\item[\expr{(define-fun $f$ (($x_1$ $\tau_1$) $\cdots$ ($x_n$ $\tau_n$)) $\tau$ $t$)}]
with $n \geq 0$ and $t$ not containing $f$ is 
semantically equivalent to the command sequence

\begin{tabular}{l}
\expr{(declare-fun $f$ ($\tau_1$ $\cdots$ $\tau_n$) $\tau$)} \\
\expr{(assert (forall (($x_1$ $\tau_1$) $\cdots$ ($x_n$ $\tau_n$)) 
       (= ($f$ $x_1$ $\cdots$ $x_n$) $t$))}.
\end{tabular}

%but differs in that $f$ is considered a definition and will be expanded in
%solver output if the \attr{expand-definitions} option is set to \ter{true}.
Note that the restriction on $t$ prohibits recursive or mutually recursive 
definitions, 
which are instead provided by \ter{define-fun-rec} and \ter{define-funs-rec}.
The command reports an error if a function symbol with name $f$ is already 
present in the current signature or if the argument $t$ is not a well-sorted term 
of sort $\tau$ with respect to the current signature extended 
with the sort associations $(x_1:\tau_1),\; \ldots,\; (x_n:\tau_n)$.
\smallskip

%%% define-const %%%
%
\new{
\item[\expr{(define-const $f$ $\tau$ $t$)}] has the same effect as
\expr{(define-fun $f$ () $\tau$ $t$)}.
\smallskip
}

%%% define-funs-rec %%%
%
\item[\expr{(define-funs-rec ($d_1$ $\cdots$ $d_m$) ($t_1$ $\cdots$ $t_m$))},]
where $m > 0$ and for $i=1,\ldots,m$,
$d_i$ has the form
\begin{center}
\expr{($f_i$ (($x_{i,1}$ $\tau_{i,1}$) $\cdots$ ($x_{i,n_i}$ $\tau_{i,n_i}$)) $\tau_i$)}
\end{center}
with 
$n_i \geq 0$ and $f_1, \ldots, f_m$ pairwise distinct,
is semantically equivalent to the command sequence 
\smallskip

{\small
\begin{tabular}{l}
\expr{(declare-fun $f_1$ ($\tau_{1,1}$ $\cdots$ $\tau_{1,n_1}$) $\tau_1$)} 
\\
\hspace{10em} \vdots 
\\
\expr{(declare-fun $f_m$ ($\tau_{m,1}$ $\cdots$ $\tau_{m,n_m}$) $\tau_m$)} 
\\
\expr{(assert (forall (($x_{1,1}$ $\tau_{1,1}$) $\cdots$ ($x_{1,n_1}$ $\tau_{1,n_1}$)) 
       (= ($f_1$ $x_{1,1}$ $\cdots$ $x_{1,n_1}$) $t_1$))}
\\
\hspace{16em} \vdots 
\\
\expr{(assert (forall (($x_{m,1}$ $\tau_{m,1}$) $\cdots$ ($x_{m,n_m}$ $\tau_{m,n_m}$)) 
       (= ($f_m$ $x_{m,1}$ $\cdots$ $x_{m,n_m}$) $t_m$))} .
\end{tabular}
}
\smallskip

This command can be used to define multiple functions recursively,
in particular, mutually recursively.\endnote{
Similar to \ter{define-fun}, while strictly not needed, 
\ter{define-funs-rec} provides a more structured way to define functions axiomatically,
as opposed to introducing a new function with \ter{declare-fun} and then 
providing its definition with \ter{assert} and a quantified formula.
This gives an SMT solver the opportunity to easily recognize 
function definitions and possibly process them specially.
}
Mutual recursion is possible since each term $t_i$ can contain any applications 
of $f_1, \ldots, f_m$.  

Note that, according to the semantics above, \ter{define-funs-rec} imposes 
no requirements that each $f_i$ be terminating
(a meaningless notion in our context) or even well-defined.\footnote{%
In fact, it is even possible, although certainly not desirable, 
to have a definition like
\expr{(define-funs-rec ((f ((x Bool)) Bool)) ((not (f x))) )},
which makes the set of formulas in the context unsatisfiable.
}
The only requirement is on the well-sortedness of the definitions.

The command reports an error if for any $i \in \{1,\ldots,m\}$ 
a function symbol with name $f_i$ is already present in the current signature 
or if $t_i$ is not a well-sorted term of sort $\tau_i$ with respect 
to the current signature extended with the sort associations 
$(f_1:\tau_{1,1}\; \cdots\; \tau_{1,n_1}\tau_1)$, \ldots,
$(f_m:\tau_{m,1}\; \cdots\; \tau_{m,n_m}\tau_m)$
and
$(x_{i,1}:\tau_{i,1}),\; \ldots,\; (x_{i,n_i}:\tau_{i,n_i})$.

\new{
Note that, as for \ter{declare-fun}, 
for each $i$, $\tau_{i,1}, \ldots, \tau_{i,n_i}$, $\tau_i$ may contain sort parameters.
}

\smallskip

%%% define-fun-rec %%%
%
\item[\expr{(define-fun-rec $f$ (($x_1$ $\tau_1$) $\cdots$ ($x_n$ $\tau_n$)) $\tau$ $t$)}]
has the same effect as
\begin{center}
\expr{(define-funs-rec (($f$ (($x_1$ $\tau_1$) $\cdots$ ($x_n$ $\tau_n$)) $\tau$)) ($t$))}
\end{center}
It provides a simpler syntax to define individual recursive functions.

\end{description}


%-------------------------------------------------------------------------------
\subsection{Asserting and inspecting formulas} \label{sec:assert}
%-------------------------------------------------------------------------------

\begin{description}
%%% assert %%%
%
\item[\expr{(assert $t$)}]
where $t$ is a well-sorted formula 
(i.e., a well-sorted element of \nter{term} of sort \ter{Bool}),
adds $t$ to the current assertion level.
The well-sortedness requirement is with respect to the current signature.

\begin{newver}
If any subterm of $t$ has a polymorphic sort
and $u_1$,\ldots, $u_n$ lists all the sort parameters in the sorts of such terms 
then, at the time a later \ter{check-sat} command is executed,
the assertion stands for all the (possibly infinitely-many) assertions of the form 
\expr{(assert $t'$)} where $t'$ is obtained from $t$ by instantiating
$u_1$,\ldots, $u_n$ with \emph{all and only} the monomorphic sorts 
in the signature at that time.
For example, consider the following script.

\begin{lstlisting}
...
(declare-sort-parameter A)
(declare-sort-parameter B)
(declare-fun is-const ((Array A B)) Bool)
(assert (forall ((a (Array A B))) 
  (= (is-cont a) 
     (forall ((i A) (j A)) (= (select a i) (select a j))))))
...
(check-sat)
\end{lstlisting}
When \ter{check-sat} is executed, 
the set of asserted formulas will include all instantiations of the assertion above 
where the parameters \ter{A} and \ter{B} 
have been replaced by the monomorphic sorts available at that point.
\end{newver}

Instances of this command of the form \expr{(assert (!~$t$ \attr{named} $f$))}, 
where the asserted formula $t$ is given a label $f$, 
have the additional effect of adding $t$ to the formulas tracked 
by the commands \ter{get-assignment} and \ter{get-unsat-core}, as explained later.

\begin{newver}
\begin{remark}
Consider the commands
\begin{lstlisting}
(declare-sort-parameter A)
(assert (forall ((x A) (y A) (z A)) (or (= x y) (= x z))))
\end{lstlisting}
%
The assertion states that the domain of every sort has cardinality at most two.
Based on the definition of \ter{assert} given above, the satisfiability 
of the asserted formula depends on the set of theory sorts available at the time 
the \ter{check-sat} is executed.
For example, 
the formula is satisfiable if the only available sort is \ter{Bool} 
but it is unsatisfiable if the available sorts include theory sorts, 
such as \ter{Int}, denoting sets of larger cardinality.
Note that this semantic is \emph{not} aligned with that of other logics
with type polymorphism, where sort/type parameters are interpreted as subsets 
of a large universe of individuals.\footnote{\new{
In such logics, the formula above would always be unsatisfiable, independently 
of the signature.
}}
In contrast, sort parameters in SMT-LIB are effectively \emph{schematic} variables 
standing for monomorphic sorts.\endnote{\new{
The decision to adopt this semantics is mostly a pragmatic one as it easier 
for SMT solvers to support sort parameters that range over sorts.
Note that because of the syntactic restriction to rank-1 polymorphism
(all sort parameters are implicitly universal quantified),
the SMT-LIB semantics is sound with respect to those where type parameters 
range over sets: every formula unsatisfiable under the former semantics is 
unsatisfiable under the latter as well. 
}}
\end{remark}

\begin{remark}
Consider the commands below, stating that every sort has a domain of size one.
\begin{lstlisting}
(declare-sort-parameter A)
(assert (forall ((x A) (y A)) (= x y)))
\end{lstlisting}
%
Note that asserting the negation of the formula above \emph{does not} have 
the effect of stating that there is a sort whose domain has cardinality 
greater than one.\footnote{
All sorts have non-empty domains in SMT-LIB.  
}
\begin{lstlisting}
(declare-sort-parameter A)
(assert (not (forall ((x A) (y A)) (= x y))))
\end{lstlisting}

Instead, the two commands above state that 
\emph{every} sort has a domain of cardinality greater than one.
Properly negating the original assertion can be done 
by using a fresh sort \emph{constant}, as in

\begin{lstlisting}
(declare-sort U 0)
(assert (not (forall ((x U) (y U)) (= x y))))
\end{lstlisting}

In general, this means that, when polymorphic sorts are involved,
special care must be taken to ensure that the formula asserted matches the intent.
This is especially true when checking, for example, the validity 
of an implication \expr{(=> $p$ $q$)} 
by asserting $p$ and \expr{(not $q$)}, and then checking for satisfiability.
\end{remark}
\end{newver}

\smallskip

%%% get-assertions %%%
%
\item[\expr{(get-assertions)}]
causes the solver to print the current set of all asserted formulas
as a sequence of the form \expr{($f_1$ $\cdots$ $f_n$)}.
Each $f_i$ is a formula syntactically identical, modulo whitespace, 
to one of the formulas entered with an \ter{assert} command and 
currently in the context.
Solvers are not allowed to print formulas equivalent to or derived from 
the asserted formulas.\endnote{
The motivation is to enable interactive users to see easily (exactly)
which assertions they have asserted, without having to keep track 
of that information themselves.
}

The command can be issued only if the \attr{produce-assertions} option,
which is set to \ter{false} by default, is set to \ter{true} 
(see Section~\ref{sec:options}).

\end{description}


%-------------------------------------------------------------------------------
\subsection{Checking for satisfiability} \label{sec:checking-for-sat}
%-------------------------------------------------------------------------------

\begin{description}

%%% check-sat %%%
%
\item[\expr{(check-sat)}] has the same effect as \expr{(check-sat-assuming ())}.
\smallskip

% Consider we have such a definition
% (define-const p (forall ((l (list alpha))) (>= (length l) 0)))
% What does it mean to have
% (check-sat-assuming (p q))
% --->
% (assert (par (alpha) (forall ((l (list alpha))) (>= (length l) 0)))) ?
% And now
% (check-sat-assuming ((not p) q))
% --->
% (assert (not p))
% (check-sat)
% The definition above is something like this
% (assert (par (alpha) (= p (forall ((l (list alpha))) (>= (length l) 0)))))
% But we cannot easily distribute the quantifier inside
% the above assertion is equivalent to
% (assert (par (alpha) (=> p (forall ((l (list alpha))) (>= (length l) 0)))))
% (assert (par (alpha) (<= p (forall ((l (list alpha))) (>= (length l) 0)))))
% (assert (=> p (par (alpha) (forall ((l (list alpha))) (>= (length l) 0))))))
% (assert (or p (par (alpha) (not (forall ((l (list alpha))) (>= (length l) 0)))))))

% (check-sat-assuming (p q))

% (assert (and p q))


%%% check-sat-assuming %%%
%
\item[\expr{(check-sat-assuming ($a_1$ $\cdots$ $a_n$))}] where $n \geq 0$
and $a_1, \cdots, a_n$ are terms of sort \ter{Bool},
%
instructs the solver to check whether the conjunction of all the formulas 
in the current context \emph{and} the \define{assumptions} $a_1, \cdots, a_n$
is satisfiable in the extension of the current logic with all the current 
user-declared and user-defined symbols.
The assumptions $a_1, \ldots, a_n$ must be formulas, i.e., terms of sort
\ter{Bool}.\endnote{
The motivation for having this command is that it corresponds to 
a common usage pattern with SMT solvers which can be implemented 
considerably more efficiently than the general stack-based mechanism needed 
to support \ter{push} and \ter{pop} commands.
%The restriction of assumptions to Boolean constants or negated constants only 
%is also motivated by efficiency concerns since user-defined constants are 
%associated with formulas that have already been internally simplified 
%at definition time. 
}

Conceptually, this command asks the solver to search for a model of the logic 
that satisfies all the currently asserted formulas as well as 
the current assumptions.
When it has finished attempting to do this, the solver should reply 
on its regular output channel (see Section~\ref{sec:response}) using 
the response format defined by \nter{check\_sat\_response} 
in Figure~\ref{fig:command-responses}.
A \ter{sat} response indicates that the solver has found a model,
an \ter{unsat} response that the solver has established there is no model, and 
an \ter{unknown} response that the search was inconclusive---because of 
resource limits, solver incompleteness, or other reasons.
%
On reporting \ter{sat} or \ter{unknown} the solver should move 
to \mode{sat} mode---and then respond to \ter{get-assignment}, \ter{get-model}, and
\ter{get-value} commands provided that the corresponding enabling option is set 
to \ter{true}.
On reporting \ter{unsat}, it should move to \mode{unsat} mode---and then respond
to \ter{get-proof}, \ter{get-unsat-assumptions}, and 
\ter{get-unsat-core} commands
provided that the corresponding enabling option is set to \ter{true}.

Regardless of how it is implemented internally, 
a \ter{check-sat-assuming} command should preserve the current context 
in the sense that at the end of the command's execution the context should be 
the same as it was right before the execution.

Note that a \ter{check-sat-assuming} command can be issued also 
when the solver is already in \mode{sat} or \mode{unsat} mode
(in this case, the context is necessarily the same as for the previous check command).
However, it is possible for the solver to switch from \mode{sat} 
to \mode{unsat} mode or vice versa if the latest command has 
a different set of assumptions from the previous one.

\end{description}


%-------------------------------------------------------------------------------
\subsection{Inspecting models} 
%-------------------------------------------------------------------------------

The next three commands can be issued only when the solver is 
in \mode{sat} mode and provide information related 
to the most recent check command.
In that case, the solver will have identified a model $\str A$ 
(as defined in Section~\ref{sec:models}) of the current logic,
and produces responses with respect to that same model until 
it receives the next check command or it exits the \mode{sat} mode,
whichever comes first.
The model $\str A$ is required to satisfy all currently asserted formulas 
and current assumptions only if the most recent check command
reported \ter{sat}.\endnote{
SMT solvers are incomplete for certain logics, typically those that include
quantified formulas.  
However, even when they are unable to determine whether the set $\Gamma$ 
of all assertions and assumptions is satisfiable or not, 
SMT solvers can typically compute a model 
for a set $\Gamma'$ of formulas that is entailed by $\Gamma$ in the logic.
Interpretations in this model are often useful to client applications 
even if they are not guaranteed to come from a model of $\Gamma$.
}

The internal representation of the model $\str A$ is not exposed by the solver.
Similarly to an abstract data type, the model can be inspected only 
through the three commands below.
As a consequence, it can even be partial internally and extended as needed 
in response to successive invocations of some of these commands.\footnote{%
In that case, of course, the solver has to be sure 
that its partial model can be indeed extended as needed.
}

\begin{description}

%%% get-value %%%
%
\item[\expr{(get-value ($t_1$ $\cdots$ $t_n$))}]
where $n > 0$ and each $t_i$ is a well-sorted closed quantifier-free term
\new{of monomorphic sort}, 
returns for each $t_i$ a value term $v_i$\footnote{%
Recall that value terms are particular ground terms defined in a logic 
for each sort (see Subsection~\ref{sec:logic-decl}).  
} 
that is equivalent to $t_i$ in the current model $\str A$ (see above).
Specifically, $v_i$ has the same sort as $t_i$, and 
$t_i$ is interpreted the same way as $v_i$ in $\str A$.
The values are returned as a sequence of pairs of the form 
\expr{(($t_1$ $v_1$) $\cdots$ ($t_n$ $v_n$))}.
The terms $v_1,\ldots,v_n$  are allowed to contain symbols
not in the current signature only if they are abstract values, 
i.e., constant symbols starting with the special character \ter{@}.\endnote{
Abstract values are useful for reporting model values in logics containing  
for example the theory of arrays (see Figure~\ref{fig:ArraysEx}). 
For instance, a solver may specify the content of an array \ter{a} of sort
\ter{(Array Int Int)} at positions 0--2 by returning the expression
\begin{center}
\expr{(define-fun a () 
(store (store (store (as @array1 (Array Int Int)) 0 0) 1 2) 2 4))} .
\end{center}
The elements of \ter{a} with index outside the 0--2 range are left unspecified 
because the array \ter{@array1} itself is left unspecified.
}
Since these are solver-defined, their sort is not known to the user.
Therefore, additionally, each occurrence of an abstract value 
$a$ of sort $\sigma$ in $v_1,\ldots,v_n$ has to be contained in a term 
of the form \expr{(as $a$ $\sigma$)} which makes the sort explicit.

Note that the returned abstract values are used only to express 
information about the current model $\str A$.
They cannot be used in later \ter{assert} commands 
since they are neither theory symbols nor user-defined ones.
However, they can be used in later \ter{get-value} commands 
on the same model.

There is no requirement that different permutations 
of the same set of \ter{get-value} calls produce the same value 
for the input terms.
The only requirement is that syntactically different values 
of the same sort returned by the solver have different meaning 
in the model.\footnote{%
So, for instance, in a logic of rational numbers and values of the form
\expr{(/ $m$ $n$)} and \expr{(/ (- $m$) $n$)} with $m,n$ numerals, 
the solver cannot use both the terms \expr{(/ 1 3)} and \expr{(/ 2 6)}
as output values for \ter{get-value}.
}

The command can be issued only if the \attr{produce-models} option,
which is set to \ter{false} by default, is set to \ter{true} 
(see Section~\ref{sec:options}).
\smallskip

%%% get-assignment %%%
%
\item[\expr{(get-assignment)}]
can be seen as a light-weight and restricted version of \ter{get-value} 
that asks for a truth assignment for a selected set 
of previously entered formulas.\endnote{
Since it focuses only on preselected, Boolean terms, 
\ter{get-assignment} can be implemented much more efficiently than 
the more general \ter{get-value}.
}

The command returns a sequence of the form 
\expr{(($f_1$ $b_1$) $\cdots$ ($f_n$ $b_n$))} with $n \geq 0$.
A pair \expr{($f_i$ $b_i$)} is in the returned sequence if and only if
$f_i$ is the label of a (sub)term of the form 
\expr{(!~$t_i$ \attr{named} $f_i$)} in the context, 
with $t_i$ a closed quantifier-free term of sort \ter{Bool},
and 
$b_i$ is the value (\ter{true} or \ter{false}) 
that $t_i$ has in the current model $\str A$.

The command can be issued only if the \attr{produce-assignments} option, 
which is set to \ter{false} by default, is set to \ter{true} 
(see Section~\ref{sec:options}).  
\smallskip

\newpage

%%% get-model %%%
%
\item[\expr{(get-model)}]
%\ctrem{We may need to add \ter{declare-sort-parameter}.
%For this reason,
%we may as well allow the declaration of abstract (sort and function) symbols with \ter{declare-sort} and \ter{declare-fun}.
%} 
returns a \define{model}, 
a list \expr{($d_1$ $\cdots$ $d_k$)} of definitions specifying \emph{all} and only the current user-declared 
function symbols $\{g_1, \ldots, g_m\}$ in the current model $\str A$
%\ctrem{We should allow also declarations of abstract symbols}.
The interpretation of each symbol is provided in exactly one of the definition 
$d_1$,\dots $d_k$.  The define commands $d_1$,\dots $d_k$ have one of the following forms:\endnote{
The rationale for providing the interpretation of a function symbol
as a define command is that 
$(i)$ the syntax of such commands is general enough to be able to express
a large class of functions symbolically in the language of the current logic
(possibly augmented with abstract values), and
$(ii)$ in principle the user could use the provided definition as is in later
interactions with the solver---as long as the original symbol's declaration 
is no longer in the current context as a consequence of a system restart, 
or a reset or a pop operation.
}
%
\begin{itemize}
\item
\expr{(define-fun $f$ (($x_1$ $\sigma_1$) $\cdots$ ($x_n$ $\sigma_n$)) $\sigma$ $t$)}

where
$n \geq 0$, $f$ has rank $\sigma_1 \cdots \sigma_n \sigma$,
$t$ is a term not containing $f$, and the formula
\smallskip

\expr{
\begin{tabular}{l}
(forall (($x_1$ $\sigma_1$) $\cdots$ ($x_n$ $\sigma_n$)) 
       (= ($f$ $x_1$ $\cdots$ $x_n$) $t$))
\end{tabular}
}
\smallskip

is well-sorted and satisfied by $\str A$.
The term $t$ is expected, although not required, to be a value 
when $f$ is a constant (i.e., when $n = 0$).
\smallskip

\item
\expr{(define-funs-rec  
  (($f_1$ (($x_{1,1}$ $\sigma_{1,1}$) $\cdots$ ($x_{1,n_1}$ $\sigma_{1,n_1}$)) $\sigma_1$) $\cdots$
\\
\phantom{(define-funs-rec (}($f_p$ (($x_{p,1}$ $\sigma_{p,1}$) $\cdots$ ($x_{p,n_p}$ $\sigma_{p,n_p}$)) $\sigma_p$))
 ($t_1$ $\cdots$ $t_p$))} 

where $n_i > 0$ for $i=1,\ldots,p$, and the formula
\smallskip

\expr{
\begin{tabular}{l@{\ }l}
(and & (forall (($x_{1,1}$ $\sigma_{1,1}$) $\cdots$ ($x_{1,n}$ $\sigma_{1,n_1}$)) 
       (= ($f_1$ $x_{1,1}$ $\cdots$ $x_{1,n_1}$) $t_1$))
\\
& \ \vdots
\\
& (forall (($x_{p,1}$ $\sigma_{p,1}$) $\cdots$ ($x_{p,n_p}$ $\sigma_{p,n_p}$)) 
          (= ($f_p$ $x_{p,1}$ $\cdots$ $x_{p,n_p}$) $t_p$)))
\end{tabular}
}
\smallskip

is well-sorted and satisfied by $\str A$.
\smallskip

\item
\expr{(define-fun-rec $f$ (($x_1$ $\sigma_1$) $\cdots$ ($x_n$ $\sigma_n$)) $\sigma$ $t$)} 
where $n > 0$ and the formula 
\smallskip

\expr{
\begin{tabular}{l}
(forall (($x_1$ $\sigma_1$) $\cdots$ ($x_n$ $\sigma_n$)) 
          (= ($f$ $x_1$ $\cdots$ $x_n$) $t$))
\end{tabular}
}
\smallskip

is well-sorted and satisfied by $\str A$.

\end{itemize}

\begin{newver}
A user-declared or user-defined function symbol can occur in the terms $t$, $t_1$,\dots $t_p$
above only if it has been previously defined in the model or is one of the symbols
being recursively defined.
\end{newver}

Similarly to the response of \ter{get-value}, 
the terms $t,t_1,\ldots,t_p$ above are allowed to contain symbols
not in the current signature only if they are abstract values.
Morevever, each occurrence of an abstract value  $a$ of sort $\sigma$ 
in $t,t_1,\ldots,t_p$ has to be contained in a term of the form 
\expr{(as $a$ $\sigma$)}.

Later versions of the standard may impose stronger requirements 
on the returned definitions.
For now there is only an expectation that, when possible, the solver will provide 
definitions that have a unique interpretation over the current signature.\endnote{
The current requirements on the returned definitions are rather weak.
For instance, they allow a solver to return something like
\begin{center}
\expr{(define-fun-rec f (($x_1$ $\sigma_1$) $\cdots$ ($x_n$ $\sigma_n$)) $\sigma$ 
   (f $x_1$ $\cdots$ $x_n$))} 
\end{center}
for a given $f$ of rank $\sigma_1\cdots\sigma_n\sigma$.
Similarly, for constant symbols $c$ of a sort $\sigma$ that admits 
abstract values, they allow a solver to return
\expr{(define-fun c () @a)} with \ter{@a} abstract.

The reason for such weak requirements is that stronger ones are 
currently difficult to achieve in general because of limitations 
in the expressive power of some SMT-LIB theories/logics or 
in the computational abilities of present SMT solvers.
For instance, the current theory of arrays (see Figure~\ref{fig:ArraysEx}) 
does not have enough \emph{constructor} symbols
to allow one to represent an array uniquely as a value term.
As shown in a previous note, one can use terms like 
\begin{center}
\expr{(store (store (store (as @array1 (Array Int Int)) 0 0) 1 2) 2 4))}
\end{center}
which fixes only a portion of the array.
This term has infinitely many interpretations that differ on the elements
at indices outside the 0--2 range.
Similarly, because of the progress in automated synthesis, it is conceivable 
that future solvers will be able to construct a model
where a user-declared function symbol $f$ denotes the factorial function 
over the non-negative integers.
In that case, a definition like
\begin{center}
\expr{(define-fun-rec f ((x Int)) Int (ite (= x 0) 1 (* x (f (- x 1)))))}
\end{center}
would not have a unique interpretation because it does not uniquely determine
the behavior of $f$ over the negative integers.
In contrast, the definition
\begin{center}
\expr{(define-fun-rec f ((x Int)) Int (ite (<= x 1) 1 (* x (f (- x 1)))))}
\end{center}
say, would determine a unique function over the whole set of integers.
%Synthesizing such definitions from a set of input constraints is 
%by and large beyond the abilities of current SMT solvers.
}

The command can be issued only if the \attr{produce-models} option,
which is set to \ter{false} by default, is set to \ter{true} 
(see Section~\ref{sec:options}).

\begin{newver}
\begin{remark}
With the given syntax, all functions definable in a model are necessarily monomorphic.
This means that in scripts with polymorphic function symbols, 
the returned model will be necessarily partial 
whenever the rank of a user-declared or user-defined function has infinitely many instances.
This restriction will be addressed as needed in later releases of this document.
\end{remark}
\end{newver}

\end{description}

%-------------------------------------------------------------------------------
\subsection{Inspecting proofs} 
%-------------------------------------------------------------------------------

The next three commands can be issued only when the solver is 
in \mode{unsat} mode, and provide information related to the most recent 
check command (which produced an \ter{unsat} response).
%The command \ter{get-unsat-assumptions} below can be issued only following 
%an invocation of \\
%\ter{check-sat-assuming} that reports \ter{unsat},
%without any intervening commands other than other instances of 
%\ter{get-unsat-assumptions}.
%The other two commands can be issued only following a \ter{check-sat} command  
%that reports \ter{unsat}, without intervening commands that modify the context.

\begin{description}
%%% get-unsat-assumptions %%%
%
\item[\expr{(get-unsat-assumptions)}]
returns a subset $(a_1 \cdots a_n)$ of the assumptions in the most recent
\ter{check-sat-assuming} command.
These assumptions are such that issuing the command
\expr{(check-sat-assuming ($a_1$ $\cdots$ $a_n$))}
instead would have still produced an \ter{unsat} response.
The returned set is a space-delimited list of assumptions surrounded by
parentheses and is not required to be minimal.\endnote{
The lax requirement is justified by the fact that 
the minimization problem alone is NP-hard in general.
On the other hand, it allows a solver to be compliant by just returning
the same sequence given to (the most recent) \ter{check-sat-assuming}.
}

The command can be issued only if the \attr{produce-unsat-assumptions} option,
which is set to \ter{false} by default, is set to \ter{true} 
(see Section~\ref{sec:options}).
\smallskip

%%% get-proof %%%
%
\item[\expr{(get-proof)}]
asks the solver for a proof of unsatisfiability for the set of all formulas
in the current context.  
The command can be issued only if the most recent check command had an empty set 
of assumptions.
The solver responds by printing a refutation proof 
on its regular output channel.
The format of the proof is solver-specific.\endnote{
There is, as yet, no standard SMT-LIB proof format.
}
The only requirement is that, like all responses, it be a member 
of \nter{s\_expr}.

The command can be issued only if the \attr{produce-proofs} option,
which is set to \ter{false} by default, is set to \ter{true} 
(see Section~\ref{sec:options}).
\smallskip

%%% get-unsat-core %%%
%
\item[\expr{(get-unsat-core)}]
Let $A$ be the set of assertions in the context and $B$ the set of assumptions
in the most recent call to  \expr{check-sat-assuming}
(where, as defined above, a call to \expr{check-sat} is considered a call to \expr{check-sat-assuming}
with no assumptions).
Let $A$ be further partitioned into $A_n$ and $A_u$, where
$A_n$ contains assertions that were asserted with a command of the form
\expr{(\ter{assert} (!~$t$ \attr{named} $f$))} and $A_u = A \setminus A_n$.
The names $f$ are called \define{labels}, and we call $A_n$ the set of
\define{labeled assertions} and $A_u$ the set of \define{unlabeled assertions}.
The result of this command is a sequence \expr{($f_1$ $\cdots$ $f_n$)} of
labels from the labeled assertions such that the corresponding subset
$A'_n$ of $A_n$ has the property that $A'_n \cup A_u \cup B$ is, by itself,
unsatisfiable.
Furthermore, we require that if $B'$ is the result of any call to
\expr{get-unsat-assumptions} in the same solver state, then $A'_n \cup A_u \cup
B'$ is also unsatisfiable.

Note that assumptions and unlabeled formulas are never included in the result
of \expr{get-unsat-core}.  However, it is possible to get a full set
of unsatisfiable assertions by labeling all assertions and combining the result
of \expr{get-unsat-core} with the result of \expr{get-unsat-assumptions}.
In practice, not labeling
assertions may be useful for unsat core detection purposes when the
user is sure that the set of all unlabeled assertions is satisfiable.  In such
cases, users are often only interested in the part of the unsat core coming
from additional assertions on top of this satisfiable base.
%
% PF I find the above sentence to be puzzling.  It seems to say that those
% assertions will not play a role in an unsat core anyway.
% CB I tried to clarify.  Note that this sentence was present before - I believe
% it was added by Cesare.
%\endnote{
%Unsatisfiable cores are useful for applications because they circumscribe
%the source of unsatisfiability in the asserted set.
%}
% CB I also felt that this endnote was not adding anything useful, so I have
% removed it.

%An unsatisfiable core is any subset $A \cup B$
%that is unsatisfiable by itself.
%The command can be issued only if the most recent check command had an empty set 
%of assumptions.
%, and returns a sequence
%\expr{($f_1$ $\cdots$ $f_n$)} of those labels.

%The semantics of this command's output is that the reported assertions
%\emph{together with all} the unlabeled ones in the set of all
%assertions are jointly unsatisfiable.

The command can be issued only if the \attr{produce-unsat-cores} option,
which is set to \ter{false} by default, is set to \ter{true} 
(see Section~\ref{sec:options}).

\end{description}


%-------------------------------------------------------------------------------
\subsection{Inspecting settings}
%-------------------------------------------------------------------------------

\begin{description}

%%% get-info %%%
%
\item[\expr{(get-info $f$)}] where $f$ is an element of \nter{info\_flag}
outputs solver information as specified in Section~\ref{sec:get-info}.
If a solver does not support a (standard or non-standard) flag $f$,
it just outputs \ter{unsupported}. 
\smallskip

%%% get-option %%%
%
\item[\expr{(get-option $o$)}] 
outputs the current value of a solver's option $o$
as an element of \nter{attribute\_value}.
The form of that value depends on the specific option.
More details on standard options and required behavior are provided 
in Section~\ref{sec:options}.  
If a solver does not support the setting of a standard option $o$,
the command outputs the option's default value.
For an unsupported non-predefined option the command outputs \ter{unsupported}.  
\end{description}


%------------------------------------------------------------------------
\subsection{Script information} \label{sec:script-info}
%------------------------------------------------------------------------

\begin{description}

%%% echo %%%
%
\item[\expr{(echo $s$)}] 
where $s$ is a string literal, simply prints back $s$ as is---including 
the surrounding double-quotes.\endnote{
Interjecting \ter{echo} commands in a script can help a software client know 
where the solver is in the execution of the script's commands. 
}
\smallskip

%%% set-info %%%
%
\item[\expr{(set-info $a$)}] where $a$ is an element of \nter{attribute}
has no effect on the assertion stack.
Its only purpose is to allow the insertion of structured meta information 
in a script.\endnote{
This is particularly useful for scripts that are used as benchmarks,
as \ter{set-info} can be used to store such information 
as authors, date, expected response for a check command,
difficulty level, and so on.
}
Typically then, a solver will just parse the command and do nothing with it, 
except for printing a response (\ter{success} or an error, for instance,
if the argument is not an element of \nter{attribute}).

There is only a small number of predefined \ter{set-info} attributes,
which are described below together with their possible values.
These attributes are used in particular in the official SMT-LIB benchmarks
at \href{http://www.smt-lib.org/benchmarks.shtml}{www.smt-lib.org}.

\begin{description}
\item[\attr{smt-lib-version}] \ 
possible values: a decimal.
\\
The value of this attribute is the version of SMT-LIB used by the benchmark (e.g., \ter{2.7}).
For benchmarks in the official repository a call to \ter{set-info} with this attribute can occur only as the first command of a script.

\item[\attr{source}] \ 
possible values: a string or a quoted symbol.
\\
The value of this attribute is a textual description of the benchmark's source,
containing, as appropriate, such information as
the name of person(s) who generated the benchmark;
the generation date; % with format YYYY-MM-DD;
the tool that generated it;
the intended application;
the solvers that were initially used or targeted to check the benchmarks;
references to related publications; 
any other information the benchmark author deems useful.

\item[\attr{category}] \ 
possible values: \ter{"crafted"}, \ter{"random"}, and \ter{"industrial"}.
\\
The value \ter{"crafted"} indicates that the benchmark was hand-crafted
while \ter{"random"} indicates that it was generated by a random process;
\ter{"industrial"} is reserved for everything else.\endnote{
Note that the three possible values are strings and so need to be in quotes.
The reasons for the values to be strings as opposed to symbols is historical.
} 

\item[\attr{license}] \ 
possible values: a string.
\\
This is a description of the license under which the benchmark is distributed.
It can be the actual text of the license, or the URL of a web site containing the description. 

\item[\attr{status}] \ 
possible values: \ter{sat}, \ter{unsat}, and \ter{unknown}.
\\
Each occurrence of the command \expr{(set-info :status sat)}
(respectively, \expr{(set-info :status unsat)})
indicates that the next check command in the script 
is expected to return \ter{sat} (respectively, \ter{unsat}).
More precisely, the expected value of a check command in a script is 
the one indicated by the most recent command of the form 
\expr{(set-info :status $v$)} in the script.
The value \ter{unknown} is used when the expected value is not known.\endnote{
Having an explicit \ter{unknown} value is useful for comparative evaluation 
of solvers, for example in the SMT-COMP competition.
}
\end{description}
\end{description}

\begin{remark}[\ter{set-info} and \ter{get-info} are unrelated]
Contrary to what their names might suggest,
\ter{set-info} and \ter{get-info} are not related.
The first command is used to store information about a script, 
the second to obtain solver-specific information.\endnote{
The reason for this unfortunate choice of names is historical.
It is being kept only for backward compatibility with previous versions. 
}
\end{remark}




